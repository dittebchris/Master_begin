During the first month, the student is to submit/upload a project plan to the thesis group at DTU Inside, outlining the objective of the thesis. In the project plan, the student is also to take into account the overarching learning objectives listed above. 

LEARNING OBJECTIVES:

A graduate of the MSc Eng programme from DTU:

can identify and reflect on technical scientific issues and understand the interaction between the various components that make up an issue

can, on the basis of a clear academic profile, apply elements of current research at international level to develop ideas and solve problems

masters technical scientific methodologies, theories and tools, and has the capacity to take a holistic view of and delimit a complex, open issue, see it in a broader academic and societal perspective and, on this basis, propose a variety of possible actions

can, via analysis and modelling, develop relevant models, systems and processes for solving technological problems

can communicate and mediate research-based knowledge both orally and in writing
is familiar with and can seek out leading international research within his/her specialist area.

can work independently and reflect on own learning, academic development and specialization
masters technical problem-solving at a high level through project work, and has the capacity to work with and manage all phases of a project – including preparation of timetables, design, solution and documentation


Use Bloom's taxonomy:
https://cft.vanderbilt.edu/guides-sub-pages/blooms-taxonomy/

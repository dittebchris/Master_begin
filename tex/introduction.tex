%short introductory statement and motivational
% WHY PARKINSONS?
% WHY THIS THESIS?
Parkinson's disease is a progressive neurodegenerative disorder. It manifests in tremor, rigidity, bradykinesia (slowness of movement), and later in the course of the disease, loss of postural reflexes. The disease also encompasses non-motor symptoms such as depression and sleep disorders \cite{Kalia2015}. 
% this is too detailed.



\subsection{Search Strategy}
Relevant information was found through a chain search from the literature lists of a few initial articles provided by my supervisors. Furthermore, Pubmed and Google Scholar were used to search for the terms "Parkinson's Disease" together with "clinical Review", "Hoehn Yahr MDS-UPDRS", 


%motivation
\subsection{Motivation} %maybe doesn't need to be explicit?
%Very widely used
%We must be able to detect changes! in our clinical trials
    %or at least we must know if we cannot detect the changes
%if we can tell them apart (stratify) we can better recruit people that may respond to a therapy and thus maybe get a drug that will help some of them
%we need to assess changes if we want to develop a "disease-modifying" drug!
The HY scale is the most widely used scale for disease severity of PD. However, 65\% of responders in a questionnaire sent to members of the Movement Disorder Society (MDS) did not feel that the HY scale adequately describes their patients. The strength of the scale lies in its ease and speed of application which gives an overall assessment of disease severity. As the scale progresses, PET indices of dopaminergic activity decline. Unfortunately, the scale is not truly ordinal, and the progressive HY stages seem to be too broad to precisely measure clinical changes; one minimally relevant clinical difference is likely smaller than the difference from one HY stage to the next.
All this suggests that there is a need for a short, fast questionnaire for PD, and maybe I can make one! (word this better!)
\cite{Goetz2004}

About interrater and test-retest reliability of HY: \cite{Martinez-Martin2018}



\subsection{Parkinson's Disease}
\subsubsection{Epidemiology}
The prevalence of Parkinson's disease among people age 65 years or older has been estimated at 950 per 100,000 in a review based on 12 European and US studies; among the entire population, prevalence is most commonly reported as 100 to 300 per 100,000.
The median incidence rate of the disease among persons age 65 or older is 160 per 100,000 person-years. In the general population, the median incidence rate has been estimated at 14 per 100,000 person-years \cite{Wirdefeldt2011}.

\subsubsection{Clinical Manifestations}
The disease manifests as both motor and non-motor symptoms. The four cardinal motor symptoms are tremor, rigidity, bradykinesia and, later on, postural instability. Other motor symptoms can include re-emergence of primitive reflexes, dysarthria% problems speaking
, dysphagia, and more \cite{Jankovic2008}. 
Non-motor manifestations of PD may begin many years before the clinical syndrome becomes evident. The features include constipation, REM sleep behaviour disorder, excessive daytime sleepiness, olfactory dysfunction, depression, pain, fatigue, cognitive impairment and autonomic dysfunction.%autonimic dysfunction: ex too much sweating 

As a number of different diseases manifest similarly (together called \textit{Parkinsonism}), the clinical picture of the disease is muddy. Empirical observations further suggest that at least two subtypes of Parkinson's disease exist, but are not well delimited or described \cite{Kalia2015}. 


Parkinson's Disease is diagnosed according to clinical criteria, typically the presence of cardinal motor symptoms, non-motor symptoms, absence of exclusionary symptoms and response to levodopa. However, the reliability of the criteria are not fully validated; one study found a diagnostic accuracy of the UK Parkinson’s Disease Society Brain Bank criteria of 76 to 82\% \cite{Jankovic2008}. 


\subsubsection{Neuropathology}
%THERE SHOULD BE AN IMAGE HERE A LA LeWitt2008
The motor symptoms of PD arise from the degeneration of the dopaminergic neurons in the substantia nigra pars compacta in the midbrain. The neurons that are lost project to the striatum where they release dopamine and initialize signaling pathways that terminate in the motor cortex. \cite{LeWitt2008}. Motor symptoms become evident when 80-85\% of the dopamine, or half of the nerve terminals, in the striatum are lost \cite{Wirdefeldt2011,LeWitt2008}. PD is associated with loss of neurons on many other regions of the brain, including the dorsal motor nucleus of the vagus, amygdala, and hypothalamus \cite{Kalia2015}. \par 

$\alpha$-synuclein is a member of a family of proteins expressed primarily in the substantia nigra \cite{Porth2015}. $\alpha$-synuclein is normally found in presynaptic terminals, where it mediates vesicle release \cite{Dickson2018}. In Parkinson's disease, abnormally folded $\alpha$-synuclein becomes insoluble and aggregates into inclusions in the cells. The inclusions can be located both in the neuron's cell body (called Lewy bodies) and in the processes (Lewy neurites). The Lewy pathology can be found not only in the brain, but also spinal cord and the peripheral nervous system \cite{Kalia2015}.






\subsubsection{Treatment}
There is no treatment for Parkinson's disease that slows or alters the course of the disease \cite{Kalia2015}. %To alleviate the symptoms of PD, striatal dopaminergic neurotransmission can be restored by dopamine agonists which stimulate the receptors \cite{LeWitt2008}. 
To alleviate the symptoms of PD by restoring striatal dopaminergic neurotransmission, dopamine levels in the brain can be increased. Dopamine itself cannot cross the blood-brain barrier; instead, Levodopa, which is a naturally occuring intermediate in dopamine synthesis, can be ingested. Levodopa is transported to the brain, where it is converted into dopamine. 
To improve the effectiveness of Levodopa, several classes of drugs can be given concurrently. Monoamine oxidase-B inhibitors that slow the breakdown of dopamine in the central nervous system, and peripherally acting AAAD inhibitors (e.g. carbidopa) that limit conversion to dopamine outside the nervous system, are two examples \cite{LeWitt2008}. 


%section on data
\subsection{Data} %belongs in methods?
The Critical Path for Parkinson's (CPP) consortium is a global collaboration which aims to foster data-driven research. It facilitates collaboration between scientists from pharmaceutical companies, academic institutions, and government and regulatory agencies. One of their methods of fostering research is collecting, standardizing and sharing patient data from past clinical trials around the world \cite{C-Path}. From this consortium, XXXX patients that had answered a MDS-UPDRS were gathered. The data is provided to members of the CPP consortium in the Study Data Tabulation Model (SDTM) format, which is defined by the Clinical Data Interchange Standards Consortium (CDISC) \cite{CDISC2017}. CDISC has developed guidelines for controlled terminology and standard database structure specifically for the MDS-UPDRS as a part of the questionnaire domain \cite{CDISC_mdsupdrs_2012}.
\par
The data format includes many domains. In this thesis it is primarily the questionnaire (qs) domain that will be used, together with demographics (dm) and family history (aphm) data. 


\subsubsection{MDS-UPDRS}
%There also needs to be stuf abt UPDRS because there are questionnaires answered in this format as well. 
The Movement Disorder Society-sponsored Unified Parkinson's Disease Scale (MDS-UPDRS) is a revision of the original UPDRS. The revision, published in 2008, has four sections: Non-motor experiences of daily living, motor experiences of daily living, motor examination, and motor complications. Seven questions from part one, and all 16 questions in part two are designed as a patient/caregiver questionnaire and can be completed without the investigator's input. The remaining questions require the involvement of an investigator \cite{Goetz2008}. \par
The questions are designed to be answered with a response of zero to four, which correspond to common clinical terms: 0 = normal, 1 = slight, 2 = mild, 3 = moderate, and 4 = severe. The questionnaire rates 65 items, some of which are separated into right/left and upper/lower extremity, for a total of 72 questions. Furthermore, in part three of the questionnaire, the rater is also asked to indicate the patient's Hoehn and Yahr stage \cite{Goetz2008}.


\begin{table}[h]
    \begin{tabularx}{\textwidth}{|l|X|X}
    Part & Title  & Example Questions \\\hline
    Ia   & Non-motor experiences of daily living (rater)  & Cognitive impairment, depressed mood \\
    Ib   & Non-motor experiences of daily living (questionnaire) & Sleep problems, constipation problems \\
    II   & Motor experiences of daily living & Speech, handwriting, tremor    \\
    III  & Motor examination & Finger tapping, arising from chair  \\
    IV   & Motor complications  & Time spent with dyskinesias, \newline complexity of motor fluctuations
    \end{tabularx}
    \caption{Description of the four parts of the MDS-UPDRS, as well as example questions}
    \label{tab:MDS-UPDRS}
\end{table}

\subsubsection{Hoehn and Yahr Scale}
The Hoehn and Yahr scale is a simple scale in which subjects are classified in stages one to five. The scale combines both functional disability and objective signs into a single assessment of disease severity. A widely used modification includes two more stages, 1.5 and 2.5. Both versions of the scale can be seen in Appendix \ref{app:HY}. The scale is the most common and widely used tool for assessing Parkinson's Disease severity across the globe. The strengths of the scale include its ease and speed of use, and there are significant correlations between both declining positron emission tomography (PET) indices of dopaminergic activity as well as decreasing quality of life (QoL) \cite{Goetz2004}.


\subsection{Quality of Life}
There are significant correlations between advancing HY stages and decreasing quality of life \cite{Goetz2004}.

